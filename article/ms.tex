\documentclass[useAMS,usenatbib]{mn2e}


\title[Inferring absolute stellar luminosities]{A probabilistic model for 
    inferring absolute stellar luminosities from spectra}
% Some arrangement of the following authors:
\author[Casey et al.]{A.~R.~Casey$^1$, M.~Ness$^2$, P.~Jofr\'e$^1$, 
    D.~W.~Hogg$^{2,3,4}$, T.~M\"adler$^1$, G.~Gilmore$^1$, 
    H.~W.~Rix$^2$, \newauthor A.~Ho$^2$, K.~Hawkins$^1$ \\ 
$^1$Institute of Astronomy, University of Cambridge, Madingley Road, Cambridge
    CB3 0HA, UK\\
$^2$Max-Planck-Institut f\"ur Astronomie, K\"onigstuhl 17, D-69117 Heidelberg,
    Germany\\
$^3$Center for Cosmology and Particle Physics, Department of Physics, New York
    University, 4 Washington Pl., room 424, New York, NY, 10003, USA\\
$^4$Center for Data Science, New York University, 726 Broadway, 7th Floor,
    New York, NY 10003, USA}
\begin{document}

\date{Accepted 2015 XX XX. Received 2015 October XX; in original form 2015 XX xx}

\pagerange{\pageref{firstpage}--\pageref{lastpage}} \pubyear{2015}

\maketitle

\label{firstpage}

\begin{abstract}
Measuring stellar distances is a fundamental challenge in astronomy. Although 
\textit{Gaia} is expected to revolutionise modern astrophysics, most of the 
$>$15 million stellar spectra to be obtained from wide-field surveys in the 
coming decade will not have \textit{Gaia} parallaxes. Indeed, parallaxes cannot 
be reliably determined for $\sim$99\% of the stars in the Milky Way. Therefore 
any scientific inferences that rely on knowing the positions of more than 1\% of 
Milky Way stars will continue to be limited by our inability to accurately 
measure stellar distances. `\textit{Stellar twins}' offer a novel approach to 
determining stellar distances: by identifying two stars with indistinguishable 
spectra, where one has a \textit{Hipparcos} trigonometric parallax, the distance
 to the secondary star is calculable from the difference in perceived 
 brightnesses. Crucially, unlike other methods to determine stellar distances, 
the distance precision does not degrade with increasing distance. Here we extend
 this idea with a probabilistic model to infer (amongst other quantities) 
absolute stellar luminosities directly from noisy spectra. Using FGK giant 
spectra from the APOGEE Survey and the ESO public archive, we show that 
distances can be reliably determined from ground-based apparent magnitudes and 
noisy spectra with an accuracy of XX pc, YY pc and ZZ pc for F, G and K-type 
giants, the most populous Galactic stellar tracers. We employ our probabilistic 
model to \textit{predict Gaia DR2 parallaxes} for XX FGK stars, and pledge to 
publicly announce a comparison once \textit{Gaia} parallaxes are made available,
 no matter what the result.
\end{abstract}

\begin{keywords}
\end{keywords}

\section{Introduction}

\section[]{Data}

\section[]{Toy Model?}

\section[]{Methods}

\section[]{Discussion}

\section{Conclusions}


\section*{Acknowledgments}
This research made use of Astropy, a community-developed core Python package for Astronomy \citep{astropy}.
% The New Milky Way conference?
% Streams @ ringberg conference?


\begin{thebibliography}{99}
%\bibitem[\protect\citeauthoryear{Baird}{1981}]{b1} Baird S.R., 1981,
%ApJ, 245, 208
\bibitem[Astropy Collaboration et al.(2013)]{astropy} Astropy Collaboration, Robitaille, T.~P., Tollerud, E.~J., et al.\ 2013, \aap, 558, AA33
\end{thebibliography}


\label{lastpage}

\end{document}

